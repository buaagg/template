\documentclass[
	10pt,
	twocolumn,
	a4paper,
]{article}
\usepackage{listings}

\usepackage{ctex}
\usepackage{graphicx}

\usepackage[top=0.75in, bottom=0.75in, left=0.5in, right=0.5in, landscape]{geometry}

\usepackage{fancyhdr}
\pagestyle{fancy}
\lhead{\includegraphics{beihang.png}Beihang Univ.}
\chead{} 
\rhead{\thepage} 
\lfoot{}
\cfoot{}
\rfoot{} 

\lstset{
	breaklines=true,
	basicstyle=\footnotesize\ttfamily,
	showspaces=false,
	language=C++,
	tabsize=4,
}

% \setlength{\parindent}{0pt}

\begin{document}
\section{Formula}
\begin{itemize} 
\setlength{\itemsep}{0pt}
\item Eular Formular: $|V| - |E| + |F| = 2$
\item $|A \cup B \cup C | = |A| + |B| + |C| - |A \cap B| - |A \cap C| - |B \cap C| + |A \cap B \cap C|$
\item $|\bar A \cap \bar B \cap \bar C | = |\Omega| - |A| - |B| - |C| + |A \cap B| + |A \cap C| + |B \cap C| - |A \cap B \cap C| $
\item Catalan Number: $C_n = (4n-2) / (n+1) C_{n-1}$
\item [TODO]Burnside引理
\item 记得树可以通过dfs序列化
\item Floyd算法别忘了设置dist[i][i] = 0
\item 想着比赛的时候可以打表
\item 初始化一定不要忘记
\item 提交时记得把所有的调试信息都删掉
\item 想着可以用二分的方法,把问题转化为判定问题。
\item 对于几何问题,没想法就先动手画画图,别上来就解析法。
\item 数组一定要开的足够打大,能用LL就别用int
\item 数位dp一定要写暴力check
\item 最大(极大)独立集 + 最小(极小)(点)覆盖集 = V
\item 最大(极大)团 = 补图的最大(极大)独立集
\item 二分图的最大独立集 = V - 二分图的最大匹配
\item 二分图的最大(点权)独立集 = SUM - 二分图的最佳匹配
\item 二分图的最小(边权)覆盖 = 二分图的最佳匹配
\item 二分图的最小(点权)覆盖 = 最小割(X-Y之间的边设为$\inf$)
\item 二分图的最小覆盖 = 二分图的最大匹配
\item 注意求递推式的时候可能要用到二项式定理,如求$ \Sigma_{i-1}^n{i^k k^i} (k \le 50) $
\end{itemize} 

\section{Edit Esp}
\lstinputlisting{editEsp.cc}
\section{Java Header}
\lstinputlisting{Main.java}
\section{Dancing Links精确覆盖(矩阵处理)}
\lstinputlisting{DLX-Exact-Mat.cc}
\section{Dancing Links重复覆盖(矩阵处理)}
\lstinputlisting{DLX-Repeat-Mat.cc}
\section{dinic}
\lstinputlisting{dinic.cc}
\section{costflow}
\lstinputlisting{costflow.cc}
\section{planarmincut}
\lstinputlisting{planarmincut.cc}
\section{kosaraju}
\lstinputlisting{kosaraju.cc}
\section{kmp AND exkmp}
\lstinputlisting{kmp.cc}
\section{FFT}
\lstinputlisting{FFT.cc}
\section{isprime}
\lstinputlisting{isprime.cc}
\section{rho}
\lstinputlisting{rho.cc}
\section{crt}
\lstinputlisting{crt.cc}
\section{log}
\lstinputlisting{log.cc}
\section{romberg}
\lstinputlisting{romberg.cc}
\section{halfPlane}
\lstinputlisting{halfPlane.cc}
\section{convex}
\lstinputlisting{convex.cc}
\section{给你一些线段,找到一条直线,要求穿过尽量多的线段 –O(n2logn)}
\lstinputlisting{sortByAng.cc}



leftist
\lstinputlisting{leftist.cc}
treap by HL
迪卡尔树
SA
AC自动机
Manacher
\end{document}
