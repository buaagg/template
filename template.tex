\documentclass[
	10pt,
	twocolumn,
	a4paper,
]{article}
\usepackage{listings}

\usepackage{ctex}
\usepackage{graphicx}

\usepackage[top=0.75in, bottom=0.75in, left=0.5in, right=0.5in, landscape]{geometry}

\usepackage{fancyhdr}
\pagestyle{fancy}
\lhead{\includegraphics{beihang.png}Beihang Univ.}
\chead{} 
\rhead{\thepage} 
\lfoot{}
\cfoot{}
\rfoot{} 

\lstset{
	breaklines=true,
	basicstyle=\footnotesize\ttfamily,
	showspaces=false,
	language=C++,
	tabsize=4,
}

\begin{document}

\section{Formula}
\begin{itemize} 
\setlength{\itemsep}{0pt}
\item Eular Formular: $|V| - |E| + |F| = 2$
\item $|A \cup B \cup C | = |A| + |B| + |C| - |A \cap B| - |A \cap C| - |B \cap C| + |A \cap B \cap C|$
\item $|\bar A \cap \bar B \cap \bar C | = |\Omega| - |A| - |B| - |C| + |A \cap B| + |A \cap C| + |B \cap C| - |A \cap B \cap C| $
\item Catalan Number: $C_n = (4n-2) / (n+1) C_{n-1}$
\item 记得树可以通过dfs序列化
\item Floyd算法别忘了设置dist[i][i] = 0
\item 想着比赛的时候可以打表
\item 初始化一定不要忘记
\item 提交时记得把所有的调试信息都删掉
\item 想着可以用二分的方法,把问题转化为判定问题。
\item 对于几何问题,没想法就先动手画画图,别上来就解析法。
\item 数组一定要开的足够打大,能用LL就别用int
\item treedp尽量别一个人瞎想
\item 数位dp一定要写暴力check
\item 最大(极大)独立集 + 最小(极小)(点)覆盖集 = V
\item 最大(极大)团 = 补图的最大(极大)独立集
\item 二分图的最大独立集 = V - 二分图的最大匹配
\item 二分图的最大(点权)独立集 = SUM - 二分图的最佳匹配
\item 二分图的最小(边权)覆盖 = 二分图的最佳匹配
\item 二分图的最小(点权)覆盖 = 最小割(X-Y之间的边设为$\inf$)
\item 二分图的最小覆盖 = 二分图的最大匹配
\item 注意求递推式的时候可能要用到二项式定理,如求$ \Sigma_{i-1}^n{i^k k^i} (k \le 50) $
\end{itemize} 

\section{Polya}
Burnside 引理:设 $G = \{p_1, p_2, ..., p_g\}$ 是目标集$[1, n]$ 上的置换群,$G$ 将$[1, n]$分成 $L$ 个等价类。
设 $c_1(p_k)$ 是在置换 $p_k$ 作用下不动点的个数(也就是长度为 $1$ 的循环的个数),则等价类的个数
$$L = \frac{1}{|G|} \Sigma_{i=1}^{g}{c_1(p_i)}$$

Pòlya 定理:设 $G=\{a_1, a_2, ..., a_{|G|}\}$是 $N=\{1,2,..,N\}$上的置换群,现用 $m$ 种颜色对这 $N$ 个点染色, 则不同的染色方案数为:
$$S = \frac{(m^{c_1}+m^{c_2}+ \ldots + m^{c_{|G|}})}{|G|}$$
常见置换的循环数
\begin{itemize}
	\item 旋转:$n$ 个点顺时针(逆时针)旋转 $i$ 个位置的置换,循环数为 $gcd(n, i)$ 
	\item 翻转:
	\begin{itemize}
		\item $n$ 为偶数时: 对称轴不过顶点,循环数为 $n/2$ 对称轴过顶点:循环数为 $n/2+1$
		\item $n$ 为奇数时:循环数为$(n+1)/2$
	\end{itemize}
\end{itemize}
立方体面用 k 种颜色涂色
$$ \frac{8k^2 + 12 k^3 + 3 k^4 + k^6}{24} $$

% general
\section{Edit Esp}
\lstinputlisting{editEsp.cc}
\section{Java Header}
\lstinputlisting{Main.java}

% DS
\section{ufset}
\lstinputlisting{ufset.cc}
\section{leftist}
\lstinputlisting{leftist.cc}
\section{CartesianTree}
\lstinputlisting{CartesianTree.cc}
\section{kmp \& z}
\lstinputlisting{kmp.cc}
%\section{字符串最小重复单元的重复次数}
%\lstinputlisting{LongestRepeatedSubstring.cc}
\section{manacher}
\lstinputlisting{manacher.cc}
\section{dfa}
\lstinputlisting{dfa.cc}
\section{sa}
\begin{itemize} \setlength{\itemsep}{0pt}
\item height[2..n]:height[i]保存的是lcp(sa[i],sa[i-1])
\item rank[0..n-1]:rank[i]保存的是原串中suffix[i]的名次
\end{itemize}
\lstinputlisting{sa.cc}
\section{sam}
\lstinputlisting{sam.cc}
\section{treap}
\lstinputlisting{treap.cc}

% Graph
\section{dinic}
\lstinputlisting{dinic.cc}
\section{costflow}
\lstinputlisting{costflow.cc}
\section{planarmincut}
\lstinputlisting{planarmincut.cc}
\section{kosaraju}
\lstinputlisting{kosaraju.cc}

% Math
\section{romberg}
\lstinputlisting{romberg.cc}
\section{exgcd}
\lstinputlisting{exgcd.cc}
\section{isprime}
\lstinputlisting{isprime.cc}
\section{rho}
\lstinputlisting{rho.cc}
\section{crt}
\lstinputlisting{crt.cc}
\section{log}
\lstinputlisting{log.cc}
\section{FFT}
\lstinputlisting{FFT.cc}

% Geom
\section{segcross}
\lstinputlisting{segcross.cc}
\section{halfPlane}
\lstinputlisting{halfPlane.cc}
\section{convex}
\lstinputlisting{convex.cc}
\section{convex3d}
\lstinputlisting{convex3d.cc}
\section{geom3d}
\lstinputlisting{geom3d.cc}
\section{给你一些线段,找到一条直线,要求穿过尽量多的线段 –$O(n^2logn)$}
\lstinputlisting{sortByAng.cc}

%
\section{Dancing Links精确覆盖(矩阵处理)}
\lstinputlisting{DLX-Exact-Mat.cc}
\section{Dancing Links重复覆盖(矩阵处理)}
\lstinputlisting{DLX-Repeat-Mat.cc}
\section{Barty}
\lstinputlisting{barty.cc}
%\section{AC自动机数位dp}
%\lstinputlisting{acdig.cc}
\end{document}
