\documentclass[
	10pt,
	twocolumn,
	a4paper,
]{article}
\usepackage{listings}

\usepackage{ctex}
\usepackage{graphicx}

\usepackage[top=1in, bottom=1in, left=0.75in, right=0.75in, landscape]{geometry}

\usepackage{fancyhdr}
\pagestyle{fancy}
\lhead{\includegraphics{beihang.png}Beihang Univ.}
\chead{} 
\rhead{\thepage} 
\lfoot{}
\cfoot{}
\rfoot{} 

\lstset{
	breaklines=true,
	basicstyle=\footnotesize\ttfamily,
	showspaces=false,
	language=C++,
	tabsize=4,
}

\begin{document}
\section{Formula}
Eular Formular: |V| - |E| + |F| = 2
|𝐴∪𝐵∪𝐶|=|𝐴| |𝐵| |𝐶| |𝐴∩𝐵| |𝐵∩𝐶| |𝐴∩𝐶| |𝐴∩𝐵∩𝐶|
|𝐴̅∩𝐵̅∩𝐶̅|=Ω |𝐴| |𝐵| |𝐶| |𝐴∩𝐵| |𝐵∩𝐶| |𝐴∩𝐶| |𝐴∩𝐵∩𝐶|
Catalan Number: $C_n = (4n-2) / (n+1) C_{n-1}$
Burnside引理
可以通过dfs序列化
Floyd算法别忘了设置dist[i][i] = 0
想着比赛的时候可以打表
初始化一定不要忘记
提交时记得把所有的调试信息都删掉
想着可以用二分的方法,把问题转化为判定问题。
对于几何问题,没想法就先动手画画图,别上来就解析法。
数组一定要开的足够打大,能用LL就别用int
数位dp一定要写暴力check

\section{Edit Esp}
\lstinputlisting{editEsp.cc}
\section{Java Header}
\lstinputlisting{Main.java}
\section{Dancing Links精确覆盖(矩阵处理)}
\lstinputlisting{DLX-Exact-Mat.cc}
\section{Dancing Links重复覆盖(矩阵处理)}
\lstinputlisting{DLX-Repeat-Mat.cc}

leftist
\lstinputlisting{leftist.cc}
treap by HL

kmp

exkmp

迪卡尔树

SA

AC自动机

Manacher


\end{document}
